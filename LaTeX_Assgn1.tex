\documentclass[a4paper,12pt]{article}
\begin{document}
\title{HOW TO TIDY UP A ROOM}
\author{Sserwanga Allan Bakyaita}
\date{\today}
\maketitle

\section {Start}
First and foremost, get all the equipment that you may need in the course of cleaning. Few of us have supplies to clean the bedroom actually stored in our bedroom. Gathering all supplies together before beginning the job will keep you from getting distracted and failing to complete the job.

\subsection {Sorting}
Now get all dirty clothing or dishes and put them in a hamper. You can put all the dirty clothes together and sort them during the washing itself. You can as well open the windows and curtains to let in a little light and fresh air. For the clean clothes, you refold or re-hang them as you keep on checking. Separate your dirty clothes from the clean ones. Put the dirty clothes directly in the wash or hamper, and hang or fold your clean clothes and place them in the closet. If you put your clothes in a dresser, fold them neatly so there will be more room in it for other clothes.

\subsection {Disposition}
Collect all trash and put in the dustbin. It is not the time to determine if you are ready to trash old copies of magazines or the pair of shoes. Just throw away obvious trash, not sorting through boxes and closets. Save the major trash sweep for when you have more time.
The next thing is to make the bed. Making the bed changes the look of a room. It makes a room more inviting and uncluttered looking. An unmade bed will make any room look messy no matter how clean it is. Take off the blanket and then put it back on neatly. You may also want to take the sheets off and wash them together with the blanket. You can also make your bed first to keep yourself more motivated to clean the room. If you make the bed first it will give you an area to do other work, like fold clothes, and organize papers and other items.

\subsection {Arrangement}
Take things that belong in other rooms to their rightful place. Put all the items that don't belong in your room in a laundry basket or box and then go around the house putting the items in the right spot. For example, you might have to take out a sibling's toy, teddy or blanket that belongs in their room or put a book away in the living room. So pick up all the misplaced items on the floor, bed, desk, etc. Put them in a basket, box, or bag. These are items that belong in another room in the house. Don't try to take them back one at a time. Just put them all in one location for now and move on.

\section {End}
Finally, dust, sweep and mop. Dusting will add an extra clean edge without taking a lot of time. Get a wet rag, or use a paper towel and the appropriate type of cleaning solution and wipe down greasy, dirty, or dusty surfaces.

\end{document}

